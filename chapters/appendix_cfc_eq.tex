\section{The Extrinsic Curvature $K_{ij}$}
The Extrinsic Curvature arise in the gravitational source terms for conserved variable $\tau$, see equation \eqref{eq:tau_source}.
Here we explicitly express the extrinsic curvature $K_{ij}$ in different flat background spacetime.

\subsubsection*{Cartesian Coordinate}
In this case, the line element $ds^2$ can be expressed as
\begin{equation}
 ds^2 = \psi^4 ( dx^2 + dy^2 + dz^2 ).
\end{equation}
The non-vanishing extrinsic curvature $K_{ij}$ are:
\begin{align}
	&K^{xx} = \frac{\gamma^{xx}}{3 \alpha} \left( 2 \frac{\partial \beta^x}{\partial x} - \frac{\partial \beta^y}{\partial y} - \frac{\partial \beta^z}{\partial z} \right)  ,\\
	&K^{yy} = \frac{\gamma^{yy}}{3 \alpha} \left( - \frac{\partial \beta^x}{\partial x} + 2 \frac{\partial \beta^y}{\partial y} - \frac{\partial \beta^z}{\partial z} \right)  ,\\
	&K^{zz} = \frac{\gamma^{zz}}{3 \alpha} \left( - \frac{\partial \beta^x}{\partial x} - \frac{\partial \beta^y}{\partial y} + 2 \frac{\partial \beta^z}{\partial z} \right) , \\
	K^{yx} = &K^{xy} = \frac{1}{2 \alpha} \left( \gamma^{xx} \frac{\partial  \beta^y}{\partial x} + \gamma^{yy} \frac{\partial  \beta^x}{\partial y} \right) ,\\
	K^{zy} = &K^{yz} = \frac{1}{2 \alpha} \left( \gamma^{yy} \frac{\partial  \beta^z}{\partial y} + \gamma^{zz} \frac{\partial  \beta^y}{\partial z} \right) ,\\
	K^{xz} = &K^{zx} = \frac{1}{2 \alpha} \left( \gamma^{zz} \frac{\partial  \beta^x}{\partial z} + \gamma^{xx} \frac{\partial  \beta^z}{\partial x} \right) .
\end{align}

\subsubsection*{Cylindrical Coordinate}
In this case, the line element $ds^2$ can be expressed as
\begin{equation}
 ds^2 = \psi^4 ( dR^2 + dz^2 + R^2 d\varphi^2 )
\end{equation}
The non-vanishing extrinsic curvature $K_{ij}$ are:
\begin{align}
	&K^{RR} = \frac{\gamma^{RR}}{3 \alpha} \left( 2 \frac{\partial \beta^R}{\partial R}  - \frac{\partial \beta^z}{\partial z} - \frac{\partial \beta^\varphi }{\partial \varphi} - \frac{\beta^R}{R} \right) ,\\
	&K^{zz} = \frac{\gamma^{zz}}{3 \alpha} \left( - \frac{\partial \beta^R}{\partial R} + 2 \frac{\partial \beta^z}{\partial z} - \frac{\partial \beta^\varphi }{\partial \varphi} - \frac{\beta^R}{R} \right) ,\\
	&K^{\varphi \varphi} = \frac{\gamma^{\varphi \varphi}}{3 \alpha} \left( - \frac{\partial  \beta^R}{\partial R} - \frac{\partial \beta^z }{\partial z} + 2 \frac{\partial \beta^\varphi}{\partial \varphi}  + 2\frac{\beta^R}{R} \right) ,\\
	K^{Rz} = &K^{zR} = \frac{1}{2 \alpha} \left( \gamma^{RR} \frac{\partial  \beta^z}{\partial R} + \gamma^{zz} \frac{\partial  \beta^R}{\partial z} \right) ,\\
	K^{R\varphi} = &K^{\varphi R} = \frac{1}{2 \alpha} \left( \gamma^{RR} \frac{\partial \beta^\varphi }{\partial R} + \gamma^{\varphi \varphi} \frac{\partial \beta^R}{\partial \varphi}  \right) ,\\
	K^{z\varphi} = &K^{\varphi z} = \frac{1}{2 \alpha} \left( \gamma^{zz} \frac{\partial \beta^\varphi }{\partial z} + \gamma^{\varphi \varphi} \frac{\partial \beta^z}{\partial \varphi}  \right) .
\end{align}

\subsubsection*{Spherical Coordinate}
In this case, the line element $ds^2$ can be expressed as
\begin{equation}
 ds^2 = \psi^4 ( dr^2 + d\theta^2 + r^2 \sin^2 \theta d\phi^2 )
\end{equation}
The non-vanishing extrinsic curvature $K_{ij}$ are:
\begin{align}
	&K^{rr} = \frac{\gamma^{rr}}{3 \alpha} \left( 2 \frac{\partial \beta^r}{\partial r}  - \frac{\partial \beta^\theta}{\partial \theta} - \frac{\partial \beta^\phi }{\partial \phi} - \frac{2}{r}\beta^r - \cot \theta \beta^\theta \right) \\
	&K^{\theta\theta} = \frac{\gamma^{\theta\theta}}{3 \alpha} \left( - \frac{\partial \beta^r}{\partial r} + 2 \frac{\partial \beta^\theta}{\partial \theta} - \frac{\partial \beta^\phi }{\partial \phi} + \frac{\beta^r}{r} - \cot \theta \beta^\theta \right) \\
	&K^{\phi \phi} = \frac{\gamma^{\phi \phi}}{3 \alpha} \left( - \frac{\partial  \beta^r}{\partial r} - \frac{\partial \beta^\theta }{\partial \theta} + 2 \frac{\partial \beta^\phi}{\partial \phi}  + \frac{\beta^r}{r}  + 2 \cot \theta \beta^\theta \right) \\
	K^{r\theta} = &K^{\theta r} = \frac{1}{2 \alpha} \left( \gamma^{rr} \frac{\partial  \beta^\theta}{\partial r} + \gamma^{\theta\theta} \frac{\partial  \beta^r}{\partial \theta} \right) \\
	K^{r\phi} = &K^{\phi r} = \frac{1}{2 \alpha} \left( \gamma^{rr} \frac{\partial \beta^\phi }{\partial r} + \gamma^{\phi \phi} \frac{\partial \beta^r}{\partial \phi}  \right) \\
	K^{\theta\phi} = &K^{\phi\theta} = \frac{1}{2 \alpha} \left( \gamma^{\theta\theta} \frac{\partial \beta^\phi }{\partial \theta} + \gamma^{\phi \phi} \frac{\partial \beta^\theta}{\partial \phi}  \right) 
\end{align}


\section{\label{appendix:vector_laplician}Vectorial elliptic equations in xCFC scheme}
As mentioned in section \ref{sec:xcfc_scheme}, instead of solving coordinate-basis components of the vector fields $X^i$ or $\beta^i$, the orthonormal-basis components for vector fields are solved in the vectorial elliptic equations \eqref{eq:X} and \eqref{eq:beta}.
The expressions of these vectorial elliptic equations in orthonormal-basis form is non-trivial when we work in cylindrical coordinate or spherical coordinate.
Here we list out the relations we implemented in \texttt{Gmunu}.

\subsubsection*{Cylindrical coordinate}
We rewrite a generic vector as
\begin{align}
	&X^{\hat{R}} \equiv X^{R} ,&
	&X^{\hat{z}} \equiv X^{z} ,&
	&X^{\hat{\varphi}} \equiv R X^{\varphi}. &
\end{align}
The conformal vector Laplacian (the left hand side of equations \eqref{eq:X} and \eqref{eq:beta}) are
\begin{align}
	&(\tilde{\Delta} \bm{X})^{\hat{R}} = \tilde{\Delta} X^{\hat{R}} 
	- \frac{X^{\hat{R}}}{R^2} - \frac{2}{R^2} \frac{\partial X^{\hat{\varphi}}}{\partial \varphi}  
	+ \frac{1}{3}\frac{\partial }{\partial R} \left( \tilde{\nabla}_j  X^j \right),  \\
	&(\tilde{\Delta} \bm{X})^{\hat{z}} = \tilde{\Delta} X^{\hat{z}} 
        + \frac{1}{3}\frac{\partial }{\partial z}\left( \tilde{\nabla}_j  X^j \right), \\
	&(\tilde{\Delta} \bm{X})^{\hat{\varphi}} = \tilde{\Delta} X^{\hat{\varphi}} - \frac{X^{\hat{\varphi}}}{R^2} + \frac{2}{R^2} \frac{\partial X^{\hat{R}}}{\partial \varphi}  
        + \frac{1}{3R}\frac{\partial }{\partial \varphi}\left( \tilde{\nabla}_j  X^j \right), 
\end{align}
where Laplacian of a scalar function $u(R,z,\varphi)$ is 
\begin{equation}
	\tilde{\Delta} u = \frac{1}{R}\frac{\partial}{\partial R}\left(R\frac{\partial u}{\partial R}\right) 
+ \frac{\partial^2 u}{\partial z^2}
+ \frac{1}{R^2}\frac{\partial^2 u }{\partial \varphi^2},
\end{equation}
the divergence of the vector $\bm{X}$ is
\begin{equation}
	\tilde{\nabla}_j X^j = \frac{\partial X^{\hat{R}}}{\partial R} 
	+ \frac{\partial X^{\hat{z}} }{\partial z}
	+ \frac{1}{R}\frac{\partial X^{\hat{\varphi}} }{\partial \varphi} + \frac{X^{\hat{R}}}{R}
	,
\end{equation}
and
\begin{align}
	&\frac{\partial }{\partial R} \left( \tilde{\nabla}_j X^j \right) = \frac{\partial^2 X^{\hat{R}}}{\partial R^2} + \frac{\partial^2 X^{\hat{z}}}{\partial R \partial z} + \frac{\partial^2 X^{\hat{\varphi}}}{\partial R \partial \varphi} - \frac{1}{R^2}\frac{\partial X^{\hat{\varphi}}}{\partial \varphi} + \frac{1}{R}\frac{\partial X^{\hat{R}}}{\partial R} - \frac{X^{\hat{R}}}{R^2} ,\\
	&\frac{\partial }{\partial z} \left( \tilde{\nabla}_j X^j \right) = \frac{\partial }{\partial z} \left( \frac{\partial X^{\hat{R}}}{\partial R} + \frac{\partial X^{\hat{z}}}{\partial z} + \frac{\partial X^{\hat{\varphi}}}{\partial \varphi} \right) + \frac{1}{R}\frac{\partial X^{\hat{R}}}{\partial z} ,\\
	&\frac{\partial }{\partial \varphi} \left( \tilde{\nabla}_j X^j \right) = \frac{\partial }{\partial \varphi} \left( \frac{\partial X^{\hat{R}}}{\partial R} + \frac{\partial X^{\hat{z}}}{\partial z} + \frac{\partial X^{\hat{\varphi}}}{\partial \varphi} \right) + \frac{1}{R}\frac{\partial X^{\hat{R}}}{\partial \varphi} .
\end{align}

\subsubsection*{Spherical coordinate}
%\begin{equation}
%\begin{aligned}
%\tilde{\nabla}^2 u_\texttt{i,j,k} = 
%  &\frac{1}{r_\texttt{i,j,k}^2} \frac{1}{\tilde{\Delta} r}\left( r^2_\texttt{i+1/2,j,k} \frac{u_\texttt{i+1,j,k} - u_\texttt{i,j,k} }{\tilde{\Delta} r} - r^2_\texttt{i-1/2,j,k} \frac{u_\texttt{i,j,k} - u_\texttt{i-1,j,k} }{\tilde{\Delta} r}\right) \\
%+ &\frac{1}{r^2_\texttt{i,j,k} \sin\theta_\texttt{i,j,k}} \frac{1}{\tilde{\Delta} \theta}\left( \sin\theta_\texttt{i,j+1/2,k} \frac{u_\texttt{i,j+1,k} - u_\texttt{i,j,k} }{\tilde{\Delta} \theta} - \sin\theta_\texttt{i,j-1/2,k} \frac{u_\texttt{i,j,k} - u_\texttt{i,j-1,k} }{\tilde{\Delta} \theta}\right) \\
%+   &\frac{1}{r_\texttt{i,j,k}^2 \sin^2\theta_\texttt{i,j,k}} \frac{1}{\tilde{\Delta} \phi^2}\left( u_\texttt{i,j,k+1} - 2u_\texttt{i,j,k}- u_\texttt{i,j,k-1} \right) \\
%\end{aligned}
%\end{equation}

We rewrite a generic vector as
\begin{align}
	&X^{\hat{r}} \equiv X^{r} ,&
	&X^{\hat{\theta}} \equiv rX^{\theta} ,&
	&X^{\hat{\phi}} \equiv r\sin\theta X^{\phi}. &
\end{align}
The conformal vector Laplacian (the left hand side of equations \eqref{eq:X} and \eqref{eq:beta}) are
\begin{align}
	&(\tilde{\Delta} \bm{X})^{\hat{r}} = \tilde{\Delta} X^{\hat{r}} 
	- \frac{2}{r^2}\left[ X^{\hat{r}} + \frac{1}{\sin\theta}\frac{\partial }{\partial \theta}\left(\sin\theta X^{\hat{\theta}}\right) + \frac{1}{\sin\theta}\frac{\partial X^{\hat{\phi}}}{\partial \phi} \right] 
	+ \frac{1}{3}\frac{\partial }{\partial r} \left( \tilde{\nabla}_j  X^j \right),  \\
	&(\tilde{\Delta} \bm{X})^{\hat{\theta}} = \tilde{\Delta} X^{\hat{\theta}} 
	+ \frac{2}{r^2}\frac{\partial X^{\hat{r}}}{\partial \theta} 
        - \frac{X^{\hat{\theta}}}{r^2\sin^2\theta} 
	- \frac{2\cos\theta}{r^2\sin^2\theta}\frac{\partial X^{\hat{\phi}}}{\partial \phi} 
        + \frac{1}{3r}\frac{\partial }{\partial \theta}\left( \tilde{\nabla}_j  X^j \right), \\
	&(\tilde{\Delta} \bm{X})^{\hat{\phi}} = \tilde{\Delta} X^{\hat{\phi}} 
	- \frac{X^{\hat{\phi}}}{r^2\sin^2\theta}
	+ \frac{2}{r^2\sin\theta}\frac{\partial X^{\hat{r}}}{\partial \phi} 
	+ \frac{2\cos\theta}{r^2\sin^2\theta}\frac{\partial X^{\hat{\theta}}}{\partial \phi} 
        + \frac{1}{3r\sin\theta}\frac{\partial }{\partial \phi}\left( \tilde{\nabla}_j  X^j \right),
\end{align}
where the Laplacian of a scalar function $u(r,\theta, \phi)$ is 
\begin{equation}
	\tilde{\Delta} u = \frac{1}{r^2}\frac{\partial}{\partial r}\left(r^2\frac{\partial u}{\partial r}\right) 
+ \frac{1}{r^2 \sin\theta}\frac{\partial }{\partial \theta} \left(\sin\theta \frac{\partial u}{\partial \theta}\right)
+ \frac{1}{r^2 \sin^2\theta}\frac{\partial^2 u }{\partial \phi^2} ,
\end{equation}
the divergence of the vector $\bm{X}$ is
\begin{equation}
	\tilde{\nabla}_j X^j = \frac{1}{r^2}\frac{\partial}{\partial r}\left(r^2 X^{\hat{r}}\right) 
	+ \frac{1}{r \sin\theta} \frac{\partial }{\partial \theta} \left(\sin\theta X^{\hat{\theta}}\right)
	+ \frac{1}{r \sin\theta} \frac{\partial X^{\hat{\phi}} }{\partial \phi}
	,
\end{equation}
and
\begin{align}
	\frac{\partial }{\partial r} \left( \tilde{\nabla}_j X^j \right) = \frac{1}{r^2\sin^2\theta}\Bigg\{ &
	\sin^2\theta \left[ \frac{\partial}{\partial r} \left( r^2 \frac{\partial X^{\hat{r}}}{\partial r}\right) - 2 X^{\hat{r}} - \frac{\partial X^{\hat{\theta}}}{\partial \theta} + r \frac{\partial^2 X^{\hat{\theta}}}{\partial r \partial \theta}\right] \\ \nonumber
	& + \sin\theta \left[ \cos\theta \left( r \frac{\partial X^{\hat{\theta}}}{\partial r} - X^{\hat{\theta}} \right) - \frac{\partial X^{\hat{\theta}}}{\partial \phi} + r \frac{\partial^2 X^{\hat{\phi}}}{\partial r \partial \phi}\right]
	\Bigg\}, \\
	\frac{\partial }{\partial \theta} \left( \tilde{\nabla}_j X^j \right) = \frac{r}{r^2\sin^2\theta}\Bigg\{ &
	\sin^2\theta \left[ r\frac{\partial^2 X^{\hat{r}}}{\partial \theta \partial r} + 2 \frac{\partial X^{\hat{r}}}{\partial \theta}\right] - X^{\hat{\theta}} - \cos\theta \frac{\partial X^{\hat{\phi}}}{\partial \phi}\\ \nonumber
	& + \sin\theta \left[ \frac{\partial}{\partial \theta} \left( \sin\theta \frac{\partial X^{\hat{\theta}}}{\partial \theta}\right) + \frac{\partial^2 X^{\hat{\phi}}}{\partial \theta \partial \phi}\right]
	\Bigg\}, \\
	\frac{\partial }{\partial \phi} \left( \tilde{\nabla}_j X^j \right) = \frac{r\sin\theta}{r^2\sin^2\theta}\Bigg\{ &
	\sin\theta \left[ r \frac{\partial^2 X^{\hat{r}}}{\partial \phi \partial r} + 2 \frac{\partial X^{\hat{r}}}{\partial \phi}+ \frac{\partial^2 X^{\hat{\theta}}}{\partial \theta \partial \phi}\right] + \cos\theta \frac{\partial X^{\hat{\theta}}}{\partial \phi} + \frac{\partial^2 X^{\hat{\phi}}}{\partial^2 \phi}
	\Bigg\}.
\end{align}
